\begin{comment}
\begin{center}
2022. február 24.

Gyakorlatvezető: Dr. Ács Bernadett

Témakör: Leképezések
\end{center}
\end{comment}

\subsection{}
Hasonló-e a két mátrix?

$$
A=\left[\begin{array}{ll}
2 & 1 \\
0 & 2
\end{array}\right] \quad B=\left[\begin{array}{ll}
2 & 0 \\
0 & 2
\end{array}\right]
$$

$$
\begin{aligned}
& A \rightarrow\left|\begin{array}{cc}2-\lambda & 1 \\0 & 2-\lambda\end{array}\right|=(2-\lambda)(2-\lambda)-1=\lambda^{2}-4 \lambda+4-1=\lambda^{2}-4 \lambda+3 = 0
\end{aligned}
$$

A megoldóképlettel kiszámolva:
$\lambda_1 = 3 \qquad \lambda_2 = 1$

\vspace{2mm}
$\lambda_{1}=3$

$$
\left[\begin{array}{rr|r}
-1 & 1 & 0 \\
0 & -1 & 0
\end{array}\right] \begin{array}{rr}
-x+y=0 & -y=0 \\
y=-x & y=0
\end{array} \quad s_{1}=\left[\begin{array}{l}
x \\
y
\end{array}\right]=\left[\begin{array}{c}
-y \\
0
\end{array}\right]=p \cdot\left[\begin{array}{c}
-1 \\
0
\end{array}\right]
$$

Ha A és B hasonlók ugyanahhoz a diagonális mátrixhoz, akkor egymáshoz viszonyítva is hasonlók. A nem diagonalizálható és B diagonalizálható, tehát nem hasonló a két mátrix.

\subsection{}
$A: \mathbb{R}^{2} \rightarrow \mathbb{R}^{2} \quad$ y tengelyre tükröz

\noindent $B: \mathbb{R}^{2} \rightarrow \mathbb{R}^{2} \quad $ x tengelyre tükröz

\vspace{2mm}
\noindent $A+B$ milyen geometriai leképezés?

$$
\begin{aligned}
& A\left(\left[\begin{array}{l}x \\y\end{array}\right]\right)=\left[\begin{array}{l}-x \\y\end{array}\right] \\
& A(i)=-i=\left[\begin{array}{c}-1 \\0\end{array}\right] \quad A(\underline{j})=j=\left[\begin{array}{l}0 \\1\end{array}\right] \\
& \underline{\underline{A}}{=}[A(\underline{i}) \mid A(j)]=\left[\begin{array}{cc}-1 & 0 \\0 & 1\end{array}\right] \\
\\
& B\left(\left[\begin{array}{l}x \\y\end{array}\right]\right)=\left[\begin{array}{c}x \\-y\end{array}\right]
\end{aligned}
$$

$$
\left.\begin{array}{c}
     B(\underline{i}) = \underline{i} = \left[\begin{array}{c}
    1  \\
    0
\end{array}\right] \\
\\
     B(\underline{j}) = -\underline{j} = \left[\begin{array}{c}
    0  \\
    -1
\end{array}\right] 
\end{array}\right\} \underline{\underline{B}} = \left[\begin{array}{cc}
    1 & 0 \\
    0 & -1
\end{array}\right]
$$

$$
\underline{\underline{B \cdot A}}=\left[\begin{array}{cc}
1 & 0 \\
0 & -1
\end{array}\right]
\left[\begin{array}{cc}
-1 & 0 \\
0 & 1
\end{array}\right]=\left[\begin{array}{cc}
-1 & 0 \\
0 & -1
\end{array}\right] \\
$$

Geometriai leképezésként:
\begin{itemize}
    \item origóra tükrözés
    \item $180^{\circ}$ - os forgatás
    \item -1-gyel való szorzás
\end{itemize}

\subsection{}
Izomorfizmus - e a leképezés?

\vspace{2mm}
$A: \mathbb{R}^{2 \times 2} \rightarrow \mathbb{R}^{4}$


$A\left(\left[\begin{array}{ll}a & b \\ c & d\end{array}\right]\right)=\left[\begin{array}{l}a \\ b \\ c \\ d\end{array}\right]$

\vspace{2mm}
$\operatorname{Ker}(A)=\left\{\left[\begin{array}{ll}0 & 0 \\ 0 & 0\end{array}\right]\right\}$

\vspace{2mm}
$Im(A)=\mathbb{R}^{4} \quad \Rightarrow$ Izomorfizmus $\checkmark$

\subsection{}
(2) $B: F(\mathbb{R}, \mathbb{R}) \rightarrow \mathbb{R}^{3} \qquad$ ($F(\mathbb{R}, \mathbb{R})$ másképpen $f: \mathbb{R} \longrightarrow \mathbb{R}$)

$$
\begin{aligned}
&\end{aligned} \quad B(f)=\left[\begin{array}{l}
f(1) \\
f(2) \\
f(3)
\end{array}\right] \\
$$

$$
\begin{aligned}
& f(1)=f(2)=f(3)=0 \\
& B(f)=\underline{0}
\end{aligned}
$$

Több ilyen lehet $\Rightarrow$ nem injektív $\Rightarrow$ nem izomorfizmus $\times$

\subsection{}
Cayley - Hamilton tétel alkalmazása

$$
\begin{aligned}
&A=\left[\begin{array}{cc}
1 & -1 \\
2 & 4
\end{array}\right]\left|\begin{array}{cc}
1-\lambda & -1 \\
2 & 4-\lambda
\end{array}\right|=(1-\lambda)(4-\lambda)+2=\lambda^{2}-5 \lambda+6=P(\lambda) \\
&p(\underline{\underline{A}})= \underline{\underline{A}}^{2}-5 \cdot \underline{\underline{A}}+6 \cdot \underline{\underline{E}}=\left[\begin{array}{cc}
-1 & -5 \\
10 & 14
\end{array}\right]-\left[\begin{array}{cc}
5 & -5 \\
10 & 20
\end{array}\right]+\left[\begin{array}{ll}
6 & 0 \\
0 & 6
\end{array}\right]=\left[\begin{array}{ll}
0 & 0 \\
0 & 0
\end{array}\right] \\
& \\
&\left[\begin{array}{cc}
1 & -1 \\
2 & 4
\end{array}\right]\left[\begin{array}{cc}
1 & -1 \\
2 & 4 
\end{array}\right]= \left[ \begin{array}{cc}
    -1 & -5 \\
    10 & 14
\end{array}\right] =
\underline{\underline{A}}^{2}
\end{aligned}
$$

\vspace{2mm}
2. megoldás:
$$
\begin{aligned}
&P(\underline{\underline{A}})=(\underline{A}-2 E)(\underline{\underline{A}}-3 \underline{E})=
\left(\left[\begin{array}{cc}
1 & -1 \\
2 & 4
\end{array}\right]
-\left[\begin{array}{ll}2 & 0 \\0 & 2\end{array}\right]\right) \cdot\left(\left[\begin{array}{cc}1 & -1 \\2 & 4\end{array}\right]-\left[\begin{array}{ll}3 & 0 \\0 & 3\end{array}\right]\right) \\
\\
& \Rightarrow \quad\left[\begin{array}{cc}-1 & -1 \\2 & 2\end{array}\right]\left[\begin{array}{cc}-2 & -1 \\2 & 1\end{array}\right] = \left[\begin{array}{ll}0 & 0 \\0 & 0\end{array}\right]
\end{aligned}
$$

Inverz mátrix számítása:

$$
\begin{aligned}
& P(\underline{A})=\underline{\underline{A}}^{2}-5 \cdot \underline{\underline{A}}+6 \cdot \underline{\underline{E}}=\underline{\underline{\underline{0}}} \\
& \underline{\underline{A}}^2 - 5 \cdot \underline{\underline{A}} = -6 \cdot \underline{\underline{E}} \\
& \underline{\underline{A}}(\underline{\underline{A}} - 5 \underline{\underline{E}}) \cdot -\frac{1}{6} = \underline{\underline{E}}
\end{aligned}
$$

Mivel az inverz mátrix egyértelmű, ez a mátrix lehet csak az.

$$
\begin{aligned}
&-\frac{1}{6}\left[\left[\begin{array}{cc}
1 & -1 \\
2 & 4
\end{array}\right]-\left[\begin{array}{ll}
5 & 0 \\
0 & 5
\end{array}\right]\right]=-\frac{1}{6}\left[\begin{array}{cc}
-4 & -1 \\
2 & -1
\end{array}\right]=\left[\begin{array}{cc}
\frac{2}{3} & \frac{1}{6} \\
-\frac{1}{3} & \frac{1}{6}
\end{array}\right] = \underline{\underline{A}}^{-1}\\
\\
&{\left[\begin{array}{cc}
1 & -1 \\
2 & 4
\end{array}\right]^{-1}=\frac{1}{6} \cdot\left[\begin{array}{cc}
4 & 1 \\
-2 & 1
\end{array}\right]}
\end{aligned}
$$

