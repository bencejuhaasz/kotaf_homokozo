\begin{comment}
\begin{center}
2022. február 17.

Gyakorlatvezető: Dr. Ács Bernadett

Témakör: Bázistranszformáció
\end{center}
\end{comment}


\subsection{}
$$
\begin{aligned}
&\underline{i}, \underline{j} \rightarrow b_{1}=\left[\begin{array}{c}3 \\-2\end{array}\right]  \qquad b_{2}=\left[\begin{array}{l}1 \\4\end{array}\right] \\
\\
& S=\left[\begin{array}{cc}3 & 1 \\-2 & 4\end{array}\right]\left[\begin{array}{c} \alpha \\ \beta \end{array}\right] = \left[\begin{array}{c}3 \alpha+\beta \\-2 \alpha+4 \beta\end{array}\right]=\alpha\left[\begin{array}{c}3 \\-2\end{array}\right]+\beta\left[\begin{array}{l}1 \\4\end{array}\right] \\
\\
& S \cdot[v]_{b}=[v]_{i j} \quad / \cdot S^{-1} \text {  létezik, mert } b_{1}, b_{2}{ } \text { lin.függetlenek }\rightarrow\operatorname{det}(s) \neq 0\\
\\
& [v]_{b}=s^{-1}[v]_{i j} \quad  \quad   \\
\\
& S^{-1}=\underbrace{\frac{1}{3-4-(-2)}}_{\operatorname{det}(s)} \cdot \underbrace{\left[\begin{array}{cc}4 & -1 \\2 & 3\end{array}\right]}_{\operatorname{adj}(s)}
\end{aligned}
$$

\subsection{}

a)
$$
\begin{aligned}
&[v]_{i j}=\left[\begin{array}{c}-1 \\ 5\end{array}\right]_{i j}\rightarrow \left[v_{b}\right]=? \\
\\
&\frac{1}{14}\left[\begin{array}{cc}
    4 & -1 \\
    2 &  3
\end{array}\right] 
\left[\begin{array}{c}
    -1 \\
    5
\end{array}\right] =
\left[\begin{array}{c}
    -4 + (-5) \\
    -2 +  15
\end{array}\right] \cdot \frac{1}{14} = \left[\begin{array}{c}
    -9 \\
    13
\end{array}\right]\cdot \frac{1}{14} = [\underline{v}]_b
\end{aligned}
$$


\noindent b)
$$
\begin{aligned}
&[w]_{b}=\left[\begin{array}{c}4 \\ -1\end{array}\right]_{b}\rightarrow \left[w\right]_{ij}=? \\
\\
& \left[\begin{array}{cc}3 & 1 \\-2 & 4\end{array}\right]\left[\begin{array}{c}4 \\-1\end{array}\right]\left[\begin{array}{c}12-1 \\-8-4\end{array}\right]=\left[\begin{array}{c}11 \\-12\end{array}\right]=[\underline{w}]_{ij}
\end{aligned}
$$
\subsection{}

Régi bázis

$$
\underline{a}_{1}=\left[\begin{array}{c}
3 \\
-2 \\
-3
\end{array}\right] \quad \underline{a}_{2}=\left[\begin{array}{l}
2 \\
4 \\
3
\end{array}\right] \quad \underline{a}_{3}=\left[\begin{array}{l}
4 \\
1 \\
5
\end{array}\right]
$$

\noindent Új bázis

$$
\underline{b}_{1}=\left[\begin{array}{l}
2 \\
4 \\
3
\end{array}\right] \quad \underline{b}_{2}=\left[\begin{array}{l}
4 \\
1 \\
5
\end{array}\right] \quad \underline{b}_{3}=\left[\begin{array}{c}
3 \\
-2 \\
-3
\end{array}\right]
$$

$$
\begin{aligned}
& \underline{b}_{1}=\underline{a}_{2}=0 \cdot \underline{a}_{1}+1 \cdot \underline{a}_{2}+0 \cdot \underline{a}_{3} \rightarrow \underline{b}_{1}=\left[\begin{array}{l}0 \\1 \\0\end{array}\right]_a \\
& \underline{b}_{2}=\underline{a}_{3}=\left[\begin{array}{l}0 \\0 \\1\end{array}\right]_{a} \quad \underline{b}_{3}=\underline{a}_{1}=\left[\begin{array}{l}1 \\0 \\0\end{array}\right]_a
\end{aligned}
$$

Az áttérési mátrix:
$$
S=
\left[\begin{array}{ccc}
    0 & 0 & 1 \\
    1 & 0 & 0 \\
    0 & 1 & 0
\end{array}\right] \quad \Longrightarrow  S^{-1}=\left[\begin{array}{lll}0 & -1 & 0 \\0 & 0 & 1 \\1 & 0 & 0\end{array}\right]
$$

\subsection{}

a) Mi a leképezés mátrixa a sajátvektorokból?

\vspace{2mm}
$\mathcal{B}: \mathbb{R}^{2} \rightarrow \mathbb{R}^{2} \quad$ 

$$
\begin{aligned}
& B=\left[\begin{array}{cc}-1 & 0 \\4 & 1\end{array}\right] \quad[B]_{S_{1} S}=\left[\begin{array}{c|c}\lambda_{1} & 0 \\0 & \lambda_{2}\end{array}\right] \\
\\
& \begin{array}{cc}{\left[\begin{array}{cc}-1-\lambda & 0 \\4 & 1-\lambda\end{array}\right]=(-1-\lambda)(1-\lambda)=0}\\
& \end{array} \\
& \lambda_{1}=-1 \quad \lambda_{2}=1
\end{aligned}
$$

Mivel páronként különbözők a sajátértékek

$\rightarrow \exists$ sajátvektorokból álló bázis

$\rightarrow$ ebben felírva a B leképezés mátrixa:

$$
\left[\begin{array}{cc}
-1 & 0 \\
0 & 1
\end{array}\right]=D
$$